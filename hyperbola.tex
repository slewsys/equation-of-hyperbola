\documentclass{amsart}

\newcommand{\phat}[1]{\hat{P}_{#1}}
\newcommand{\qhat}[1]{\hat{Q}_{#1}}
\newcommand{\rhat}[1]{\hat{R}_{#1}}
\newcommand{\qsum}{Q_{12} + Q_{23} + Q_{31}}

\begin{document}
\title{An equation of the Hyperbola}
\author{A. L. Moore}
\email[A. L. Moore]{andrew.l.moore@icloud.com}
\date{April 3, 2017}
\keywords{descriptive geometry, numerical analysis}
\subjclass[1991]{Primary: 51N05, Secondary: 65D10.}
\begin{abstract}
We derive an equation, $g(x)$, of the hyperbola through three given points
$(\alpha_i, \beta_i)$ \, $i = 1, 2, 3$ with asymptotes having slopes
$m_1$ and $m_2$:
\[
  g(x) = (\frac{m_1 + m_2}{2}) k(x) \pm
  (\frac{m_1 - m_2}{2}) \sqrt{k^2(x) + 4 r} - m_1 s - m_2 t,
\]
where $k(x) = x + s + t$ and $r, s$ and $t$ are
connected to the given points by the equations:
\[
  r + (s + a_i) (t - b_i) = 0, \qquad i = 1, 2, 3.
\]
\end{abstract}
\maketitle

\section{Introduction}
Using elementary methods, we begin by solving the system:
\begin{equation}\label{E:rightHyperbola}
  r + (s + a_i) (t - b_i) = 0, \qquad i = 1, 2, 3,
\end{equation}
for $r$, $s$, and $t$ in terms of the $(a_i, b_i),$ giving:
\begin{align}
r &= a_3 \phat{12} + a_1 \phat{23} + a_2 \phat{31}, \notag\\
-s &= a_3 \qhat{12} + a_1 \qhat{23} + a_2 \qhat{31}, \notag\\
  t &= b_3 \rhat{12} + b_1 \rhat{23} + b_2 \rhat{31}, \notag
\end{align}
where, for $Q_{ij} = \frac{(a_i - a_j)^2}{b_i - b_j}$ and
$R_{ij} = a_k (b_i - b_j)$, \quad $\{i, j, k\} = \{1, 2, 3\}$
\begin{align}
\phat{ij} &= \frac{a_i^2 - a_j^2}{\qsum}, \notag\\
\qhat{ij} &= \frac{Q_{ij}}{\qsum}, \notag\\
\rhat{ij} &= \frac{R_{ij}}{\qsum}. \notag
\end{align}
For given point coordinates $(\alpha_i, \beta_i),$ \,
$i = 1, 2, 3,$ define
$\begin{pmatrix}
  a_i\\
  b_i
\end{pmatrix}
= C
\begin{pmatrix}
  \alpha_i\\
  \beta_i
\end{pmatrix},$
where
\[
  C =
    \begin{pmatrix}
      1 & -1\\
      m_1 & m_2
    \end{pmatrix} .
\]
\section{Geometric Interpretation}
Denote by $\Omega$ the
graph of~(\ref{E:rightHyperbola}),
regarding $a_i$ as an independent variable, $x,$ and $b_i$
a dependent variable, $y = f(x):$
\[
\Omega = \{(x, y) \mid y = f(x)\},
\]
and define $g(x)$ by the graph $\Gamma$ satisfying $\Gamma =  C \;
\Omega,$ i.e.
\[
  \{
  \begin{pmatrix}
    x\\
    g(x)
  \end{pmatrix}
  \}
  =
  \{
  C
  \begin{pmatrix}
    x\\
    f(x)
  \end{pmatrix}
  \} .
\]
Letting $h(x) = x - f(x),$ evidently
\[
g(x) = g \circ h \circ h^{-1}(x) = m_1 h^{-1}(x) - m_2 f \circ h^{-1}(x),
\]
which, upon substitution for $h^{-1}(x),$ yeilds:
\[
  g(x) = (\frac{m_1 + m_2}{2}) k(x) \pm
  (\frac{m_1 - m_2}{2}) \sqrt{k^2(x) + 4 r} - m_1 s - m_2 t,
\]
where $k(x) = x + s + t.$
\end{document}

%%% Local Variables:
%%% mode: latex
%%% TeX-master: t
%%% End:
